%% ******************************************
%% CONCEPTS
%%
%% Author:  Kevin Depue && Lucas Ellis (2007)
%% Purpose: Introduces opCpp concepts.
%% ******************************************

\chapter{Concepts}
\label{chap:concepts}

\begin{comment}
note: the challenge here is to get the reader (and us) to think
of these things not in terms of the implementation, but in terms 
of concepts and new ways of thinking about code.
\end{comment}

\begin{comment}
note: I think a good way of looking at opC++ concepts, is a widening
of definitions to enable capturing features from other languages.
Then look at it as giving you the ability to constrain these wide
definitions into usable and specific constructs.  Yes this is what
the approach should read like - I think we have a tendency to say
that it captures all features - but it just enables capturing some
features, and it will improve as the tool evolves.
\end{comment}

%%----------
%% Overview
%%----------
\section{ Overview }
\label{sec:concepts_overview}

This chapter outlines the concepts found in opC++.  The
first concept is a "category," a more general version
of classes in OOP.  Second, opC++ adds an "enumeration" concept,
which is a more general version of an enum.  Third, opC++
adds a modifier concept - which is a compile time version of
attributes/annotations found in other languages.  
Last we look at the ability in opC++ to define language constraints.

%%----------
%% Category
%%----------
\section{ Category }
\label{sec:concepts_category}

A Category is an OOP class which captures any number of 
language features.  We'll look at the possible features
a category can capture in this section, and what useful
features currently exist in other languages.  Specifically
we'll look at what "class" means in C\#, C++, Java, and UnrealScript.
Each of these languages target different uses - from the lower level
uses of C++ to the game-specific language Unrealscript.

Diagram of C\#, C++, Java, Unrealscript and a metaclass - the category.

\paragraph{What are language features?}
Language features add functionality at a built-in language level instead of
relying solely on collections of code.  Features can make
programming much easier - especially when specially designed for a purpose.  
Often a language feature is something that was very difficult 
to implement in a previous language.  These features do not come without costs however - if
you add a feature at a class level, and it remains unused, 
the feature can impose an unnecessary cost on the program.  Due
to these costs languages like C++ have almost no hidden costs, but
have fewer "nice" features.

The features of a class are usually entirely specific to the language
the class is implemented in.  When we describe features we usually refer
to a language specifically - such as that 'C\# has attributes' or 'Java has reflection'.  
With Categories we can pull the feature out of the language context and apply it instead to a particular
named category.  We can state that 'the class category is private by default' or that 'the windowclass category supports windowing' - this lets us talk about features independent of language.

Now we'll look at some of the features in common general purpose languages, and how they
mesh with categorization.  We'll also examine a language with a more specific purpose, the game scripting language UnrealScript.  Looking at this range of languages is important to get a feel for how
languages are becoming more project specific - we can see this with attributes in general purpose languages,
and with custom project specific languages becoming more common.

The current trend has been to move from general purpose languages to more specific languages, with
language features becoming more project focused.  

\paragraph{C++ class}
C++ provides a no frills definition of a class. It's defined as an object oriented programming construct
that encapsulates functions, data, and can inherit its type, functions and data from other classes.  C++ actually provides
two types of classes, "class" and "struct".  The differences are technically confined to their default access specifiers.
In practice you use class when working with heavy objects and struct when creating lightweight data structures.  We
can view class and struct as categories with very minor differences.

class and struct as categories.

table showing differences.

Other features in C++ classes include defining functions as polymorphic (virtual), abstract/pure-virtual, and defining them as inline.  C++ also gives good control over data layout, data sizes, and on certain compilers, data byte alignments - all useful when you need to work closely with hardware.

Example showing polymorphism, declaring functions and data.

C++ is a compile-time oriented language, so there are few dynamic typing features.  The one dynamic typing feature
standard in the language, Run Time Type Information (RTTI), is considered to be slow in practice, and is only available if a type is polymorphic (has virtual functions).

Example showing C++'s dynamic features.

\paragraph{C\# class}
Since C\# is a Just-in-time compiled language (JIT), it can do more things with its programs on demand, 
versus a compiled language like C++.  C\# classes support a number of useful features, including
attribute oriented programming, full reflection support, and some valuable new types.  Attribute oriented
programming has been proven useful as it keeps any data relevant to a member written at the declaration location.  Attributes can be used to flag particular members, add extra data to a specific member (metadata) and can be extremely
valuable when dealing with unknown classes in a library framework.

Example of attribute oriented programming in C\#.

Other design changes were made when adapting C++ like functionality into Microsofts .NET Framework.  C\#
only supports single inheritance of classes - and instead provides an alternative construct, made
popular in java, the "interface".  In general most things that are done as multiple inheritance
in C++, can be done with a meaning closer to the actual intent with interfaces.  Practically, using
interfaces means that you can attach behaviors to unrelated types, and discover whether a type implements
a behavior using the interface casting mechanism in C\#.

Example of interfaces in C\#, example of casting to an interface to discover implemented behavior.

C\# also adds the delegate type to the language.  Delegates are a more general version of the 
function pointer in C++, with much nicer syntax, and the ability to do multiple dispatch.

Example of delegate use in C\# vs function pointers in C++.

Table of important features in C\# classes.
- run-time attributes
- interfaces
- delegate

\paragraph{Java class}

Since Java is an interpretted language, it's comparatively easy to
change code after the code has been compiled.  This has made Java a popular
language for experimenting with new language features - such as Aspect Oriented
Programming (AspectJ).  As Java has evolved it has added support for attribute
oriented programming, as well as code generation based on annotations.  Using
the standard "Metadata Facility for Java" or the XDoclet project, you can generate code as 
a byproduct of using attributes.

Example of attributes and their possible implications in Java.

Java adds a special keyword only for serialization, "transient".  This
keyword tells the serialization runtime that a data member should not 
be serialized.

Table of important java class features
- attributes
- serialization support (transient)
- code generation from attributes
- hash
- to string

\paragraph{Unrealscript class}

A common language in game development is the Unreal Engine scripting language "Unrealscript".
Unlike a general programming language in a custom scripting language you always have to deal with
the interface between your scripts and main application.  Because of this boundary it uses keywords like
"event" for native-to-script transitions, and "native" for script-to-native communication.  Since
it has a specific goal in mind - gameplay programming, it also contains specific keywords that only
interact with the engine.  The "exec" keyword makes a function callable from the engine input console.
It contains class attributes - such as "ini("inifile")" which alter where settings are stored.  
It contains support for editor reflection on particular data members with another attribute syntax.

Example of an unrealscript class w/ attributes.

Unrealscript also deals with storing off default data values, which may be data or objects.  This default value
data can be read statically by script code.

Example of default instances.

As a game language UnrealScript also pays particular attention to AI and time.  Unrealscript has
support for sequenced execution of actions (latent functions), and state based switching
of functions.

Example of states in Unrealscript.

Unrealscript is an example of a language written specifically for one project.  It's feature
rich because of this, and although each feature can add a size or performance cost to the language, 
they make the language much better.

\paragraph{Abstract class}

now generalize into a definition for a category.
look at what features each language brings to the table.
give example categories that might be useful to have.

From the previous examples, we've seen that languages can differ 
greatly in terms of language features.  We see that as a language
gets more specific, it can add more features without unnecessary costs.
Also, the language of a particular class determines its features.

Table summarizing these languages and their features

In opC++ we'll be dealing with a more abstract class called "categories" which
can group language features beyond hardcoded class and struct identifiers.

%% category members
\subsection{ Declaring Members }
\label{sec:concepts_members}

what kind of members can you declare in categories?
this should maybe do an overview of c++ even.

this should have a table showing all the syntax valid within opC++ and C++

this should have a table showing all syntax valid within C++ only

this should have a table showing all syntax valid within opC++ only


%% initialization
\subsection{ initialization }
\label{sec:concepts_initialization}

In opC++, you can initialize data members directly at their definitions.
This is a looser syntax than standard C++, which can only initialize const
static integer data members.  The result of the initialization syntax can
be defined as a language feature - it does not coorespond directly to
default constructors, although it can be made to work that way.

the loose initialization rules in opcpp.

%% syntax
\subsection{ syntax }
\label{sec:concepts_syntax}

include a table detailing valid and invalid syntax
in opcpp vs c++.

this should display a table with valid/invalid non-member syntax.

\begin{table}[htb]
\centering
\begin{tabular}{|l|}
\hline 
\multicolumn{1}{|c|}{\color{blue}{Allowed Category Preprocessors}} \\
\hline 
\#if \\
\#ifdef \\
\#ifndef \\
\#elif \\
\#else \\
\#endif \\
\hline
\end{tabular}
\caption{Table showing which \cpp\ preprocessors are allowed inside categories.}
\label{table:allowed_category_preprocessors}
\end{table}

%%----------
%% Enumeration
%%----------
\section{ Enumeration }
\label{sec:concepts_enumeration}

what is an enum?

why is it useful to abstract enum into Enumeration types?

example of a generated strongly typed Enumeration.

%%----------
%% Modifiers
%%----------
\section{ Modifier }
\label{sec:concepts_modifier}

Modifiers in opC++ are similar to attributes and annotations in other languages.  However,
modifiers are a language level feature, and may observe special rules at compile time.

Modifiers are enabled for specific categories, and can be defined for data members, function
members, and category definitions.

example of modifiers in opC++

%% basic modifiers
\subsection{ basic modifiers }
\label{sec:concepts_basic_modifiers}

what does a basic modifier look like?

on data members? on functions?

example of basic modifiers in opC++

%% value modifiers
\subsection{ value modifiers }
\label{sec:concepts_value_modifiers}

what does a value modifier look like?

on data members? on functions?

example of value modifiers in opC++


%% data modifiers
\subsection{ data modifiers }
\label{sec:concepts_data_modifiers}

const, volatile

modifiers can only apply to data, functions or both

examples of data modifiers (as abstract attributes)

%% function modifiers
\subsection{ function modifiers }
\label{sec:concepts_function_modifiers}

explicit, virtual, inline

modifiers can only apply to data, functions or both

examples of function modifiers (as abstract attributes)

%% access modifiers
\subsection{ access modifiers }
\label{sec:concepts_access_modifiers}

public, private, protected

%% special modifiers
\subsection{ special modifiers }
\label{sec:concepts_special_modifiers}

uninline, inline, static

%% automatic modifiers
\subsection{ automatic modifiers }
\label{sec:concepts_automatic_modifiers}

In addition to specified modifiers, opC++ also automatically applies
a number of modifiers to declarations.  These modifiers exist in order
to allow you to filter through and constrain language syntax.  Currently
these modifiers only apply to the original statement and do not derive from
types declared elsewhere.

example of a data statement and the generated modifiers
example of a function statement and the generated modifiers

example of automatic modifiers only applying to the exact statement.

need a whole table describing what modifiers are generated when

%%----------
%% Constraints
%%----------
\section{ Constraints }
\label{sec:concepts_constraints}

Constraints in opC++ allow you to specify extra
language options for validation.  Disallow constraints
let you specify invalid modifier combinations.  Modifier
Value constraints allow you to specify what values
are valid for a modifier.

%% disallow
\subsection{ disallow }
\label{sec:concepts_disallow}

Disallow constraints may specify a custom error message.

Example: tighten back the initialization options.

Example: dont allow certain modifier combinations

Example: enforce naming conventions.

%% modifier values
\subsection{ modifier values }
\label{sec:concepts_modifier_values}

Modifier value constraints may specify a custom error message.

Example: only want float numbers in a version() modifier.

Example: only want an identifier in another modifier.

