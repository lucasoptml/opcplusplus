%% ************
%% COMMAND LINE
%% ************

%% COMMANDS
\newcommand{\cloption}[2]{{\tt #1} \\ \hspace*{4ex} #2 \\}

% command line
\chapter{Commandline}
\label{chap:commandline}

This chapter lists all the available commandline options in the \opCPP\ compiler.  An important thing to note is that if a file or directory includes spaces, it must be in quotes!  Also, separate multiple strings with commas and {\em no spaces}.\\

Example:  {\tt opcpp -oh file1.oh,"file 2.oh" -d dir1,dir2 -opobject uclass,scriptclass}

\section{Commands}

% commandline options
\cloption{-oh <files>}{Input files to be compiled.}

\cloption{-d <directories>}{Input file directories, separated by a comma.}

\cloption{-ohd <directories>}{All .oh files within these directories will be parsed.}

\cloption{-gd <directory>}{Generated files output directory.}

\cloption{-globmode}{Enables \opCPP\ glob mode.}

\cloption{-verbose}{Prints verbose output.}

\cloption{-silent}{Only prints out compiler errors and tree (if enabled).}

\cloption{-fm}{User defined function modifiers.}

\cloption{-dm}{User defined data modifiers.}

\cloption{-vdm}{User defined valued data modifiers.}

\cloption{-opobject}{User defined \opCPP\ object types.  Can only contain alphabetic charcters!}

\cloption{--opobject-remove-defaults}{Removes the default \opCPP\ object type specifiers. (\opcppK{opclass}, \opcppK{opstruct})}

\cloption{-nodebug}{Suppress debug line directives (suppress debugging redirection support).}

\cloption{--opmacro-expansion-depth}{Specifies the maximum opmacro expansion depth.}

\cloption{-altclass}{Specify alternative mappings for class categories (syntax: classprefix=category,...).}

\cloption{-altstruct}{Specify alternative mappings for struct categories (syntax: structprefix=category,...).}

