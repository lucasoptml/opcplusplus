%% **********
%% ALTERNATES
%% **********

\chapter{Alternates}
\label{chap:alternates}

\section{{\tt opinclude}}

The \opcppK{opinclude} keyword is similar to the \opcppK{#include} keyword in \cpp, except that it is used to include {\tt .oh} files.  \opCPP\ ignores the \opcppK{#include} command, but if the \opcppK{opinclude} keyword is hit, it will load and parse the file specified.  Figure \ref{fig:opinclude1} shows how to use \opcppK{opinclude}.  \opCPP\ will look for the file to be included in every directory specified on the command line and in the current working directory.

% fig:opinclude1
\begin{opcpp}[label={fig:opinclude1},caption={The \opcppK{opinclude} keyword.}]
opinclude "SomeFile.oh"
\end{opcpp}

\section{{\tt opdefine}}

See Section \ref{sec:opdefine}.

\section{{\tt opstatic}}

The \opcppK{opstatic} keyword is an {\tt opobject}-specific keyword that makes it easier to initialize static class members.  In standard \cpp, you must declare a static variable in the header file and initilize its default value in the source file.  The \opcppK{opstatic} keyword allows you to declare a static variable in the header file and initialize it in place.  Figure \ref{fig:opstatic1} shows an example of this.  Unlike modifiers (see Section \ref{chap:modifiers}), the order of keywords in an \opcppK{opstatic} declaration matters.  The \opcppK{opstatic} keyword must be the first keyword in the declaration - this may be changed in the future.  Figure \ref{fig:opstatic2} makes this clear.  Finally, it is not necessary to set an \opcppK{opstatic} variable to a value.  You can simply declare the variable as \opcppK{opstatic}, as shown in Figure \ref{fig:opstatic3}.

% fig:opstatic1
\begin{opcpp}[label={fig:opstatic1},caption={The \opcppK{opstatic} keyword.}]
/**** The static keyword in normal C++. ****/

// Header file.
class Actor
{
private:
	static bool bRenderable;
};

// Source file.
bool Actor::bRenderable = true;

/**** The opstatic replacement keyword in opC++. ****/ 

// Header file.
opclass Actor
{
private:
	opstatic bool bRenderable = true;
};
\end{opcpp}

% fig:opstatic2
\begin{opcpp}[label={fig:opstatic2},caption={The \opcppK{opstatic} keyword must come first in an \opcppK{opstatic} declaration.}]
// Invalid syntax (opstatic is not the first keyword).
opclass A
{
private:
	public opstatic int x = 3;
};

// Valid syntax (opstatic is the first keyword).
opclass A
{
private:
	opstatic public int x = 3;
};
\end{opcpp}

% fig:opstatic3
\begin{opcpp}[label={fig:opstatic3},caption={Declaring a variable \opcppK{opstatic} without initializing it.}]
// opstatic variable without initialization.
opclass A
{
private:
	opstatic public int x;
};
\end{opcpp}

